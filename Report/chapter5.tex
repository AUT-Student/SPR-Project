\chapter{جمع‌بندی و نتیجه‌گیری}
در این پروژه سه روش برای انتخاب ویژگی در مسائل دسته‌بندی متن بررسی شد: روش \lr{IGFSS} آیسال\cite{uysal2016improved}، روش \lr{MRDC} لبنی همکاران\cite{labani2018novel} و روش بر پایه الگوریتم ژنیتک غارب و همکاران\cite{ghareb2016hybrid}. ایده موجود در روش \lr{IGFSS} حول آن بود که کلاس‌های مختلف سهم برابری در تعداد ویژگی‌ها داشته باشند و همچنین به ویژگی‌‌هایی که برای شناخت عدم عضویت به یک کلاس استفاده می‌شود اهمیت داد و در اصل سهم ویژگی‌های مثبت و منفی باید از یکدیگر جدا باشد. در روش \lr{MRDC} ایده اصلی آن بود که همبستگی ویژگی‌ها با یکدیگر مورد توجه باشد و به صورت مستقل ویژگی‌ها انتخاب نشوند. نهایتا در روش بر پایه ژنتیک، از الگوریتم ژنتیک برای بهبود خروجی روش‌های انتخاب ویژگی کمک گرفته شد.
\\

با بررسی تئوری و توجه به اعدادی که در مقاله گزارش شده بود، دریافتیم که روش بر پایه ژنتیک خلاقیت بهتری دارد ولی از نظر سرعت و حافظه چندان مناسب نیست. در میان دو روش دیگر، روش \lr{IGFSS} سرعت بهتر و خلاقیت بیشتری داشته است ولی از نظر دقت در جایگاه پایین‌تری بوده است.
\\

در این پروژه تنها سه روش جدید برای بهبود انتخاب ویژگی در مسائل دسته‌بندی مطرح شد. قطعا روش‌های بیشتری را می‌توان مطالعه و بررسی کرد. هیچگاه نمی‌توان یک روش را از تمام لحاظ و برای تمام مجموعه‌های داده و متناسب از روش دیگر برتر دانست. در این شرایط باید روش‌های مختلف را برای کاربرد‌های مختلف مورد ارزیابی قرار داد و نمی‌توان تنها بر یک روش تکیه کرد؛ لذا بررسی بیشتر روش‌ها همچنان سودمند است.
\\

با بررسی همین سه روش، امکان توسعه روش‌های ترکیبی بر مبنای آن وجود دارد و خود می‌توان یک کار تحقیقاتی مجزا باشد. یعنی آنکه می‌توان ابتدا با دو روش \lr{IGFSS} و روش \lr{MRDC} یک مجموعه ویژگی اولیه ایجاد کرد و سپس با الگوریتم بر پایه ژنتیک این مجموعه را بهبود داد. همچنین می‌توان این سه روش و سایر روش‌ها را به صورت موازی اجرا کرد و برای یک مسئله خروجی را مدنظر قرار داد که دقت بهتری در مسئله دسته‌بندی داشته است. بدین شکل اگرچه حافظه و زمان بیشتری استفاده می‌شود اما دقت نهایی بالاتر خواهد بود. 