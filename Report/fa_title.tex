%% -!TEX root = AUTthesis.tex
% در این فایل، عنوان پایان‌نامه، مشخصات خود، متن تقدیمی‌، ستایش، سپاس‌گزاری و چکیده پایان‌نامه را به فارسی، وارد کنید.
% توجه داشته باشید که جدول حاوی مشخصات پروژه/پایان‌نامه/رساله و همچنین، مشخصات داخل آن، به طور خودکار، درج می‌شود.
%%%%%%%%%%%%%%%%%%%%%%%%%%%%%%%%%%%%
% دانشکده، آموزشکده و یا پژوهشکده  خود را وارد کنید
\faculty{دانشکده مهندسی کامپیوتر}
% گرایش و گروه آموزشی خود را وارد کنید
\department{}
% عنوان پایان‌نامه را وارد کنید
\fatitle{روش‌های انتخاب ویژگی برای مسائل دسته‌بندی متن}
% نام استاد(ان) راهنما را وارد کنید
\firstsupervisor{دکتر محمد رحمتی}
%\secondsupervisor{استاد راهنمای دوم}
% نام استاد(دان) مشاور را وارد کنید. چنانچه استاد مشاور ندارید، دستور پایین را غیرفعال کنید.
%\firstadvisor{نام کامل استاد مشاور}
%\secondadvisor{استاد مشاور دوم}
% نام نویسنده را وارد کنید
\name{علیرضا }
% نام خانوادگی نویسنده را وارد کنید
\surname{مازوچی}
%%%%%%%%%%%%%%%%%%%%%%%%%%%%%%%%%%
\thesisdate{بهمن 1400}

% چکیده پایان‌نامه را وارد کنید
\fa-abstract{
داده‌های متنی امروزه بخش زیادی از داده‌ها را تشکیل می‌دهند. دسته‌بدی این متون به تعدادی دسته می‌تواند کاربردهای بسیاری داشته باشد. از آنجایی که ابعاد یک متن معمولا بسیار بالاست، اعمال روش‌های دسته‌بندی متن کلاسیک روی این داده‌ها به هیچ عنوان ساده نیست. در این شرایط باید تعداد ویژگی‌ها را کاهش داد. یک راه برای کاهش تعداد ویژگی‌ها، انتخاب ویژگی‌هاست. در این پروژه سه روش جدید و کارآمد که برای انتخاب ویژگی برای مسائل دسته‌بندی متن معرفی شده‌اند معرفی خواهد شد. همچنین علاوه بر آنکه این سه روش معرفی می‌شوند، مقایسه‌ای جامع از سه روش در کنار یکدیگر ارائه خواهد شد.
}


% کلمات کلیدی پایان‌نامه را وارد کنید
\keywords{دسته‌بندی متن، انتخاب ویژگی، کاهش ابعاد و الگوریتم ژنتیک}



\AUTtitle
%%%%%%%%%%%%%%%%%%%%%%%%%%%%%%%%%%
\vspace*{7cm}
\thispagestyle{empty}
\begin{center}
\includegraphics[height=5cm,width=12cm]{besm}
\end{center}