\chapter{مفاهیم تئوری}
در این بخش قصد داریم در مورد مفاهیم تئوری که در روش‌های مورد بررسی این پروژه استفاده شده‌اند بپردازیم.


\section{روش‌های انتخاب ویژگی}

\subsection{بهره اطلاعاتی}
بهره اطلاعاتی\LTRfootnote{Information Gain}
یکی از معیارهای محبوب برای انتخاب ویژگی در مقالات است
\cite{labani2018novel}\cite{uysal2016improved}.
نحوه محاسبه این معیار برای یک کلمه در رابطه ۲-۱ آمده است.

\begin{equation}
IG(t) = -\sum_{i=1}^M P(C_i)\log{P(C_i)} + P(t)\sum_{i=1}^M P(C_i|t)\log{P(C_i|t)} + P(\bar{t})\sum_{i=1}^M P(C_i|\bar{t})\log{P(C_i|\bar{t})}
\end{equation}

در این رابطه
$IG(t)$
به معنای مقدار بهره اطلاعاتی برای کلمه
$t$
است. 
$M$
  برابر با تعداد کلاس‌ها است.
$P(C_i)$
احتمال کلاس
$C_i$
است؛ یعنی چه تعدادی از اسناد به این کلاس تعلق دارند.
$P(t)$
احتمال مربوط به کلمه
$t$
است؛ یعنی آنکه چه تعدادی از اسناد شامل این کلمه هستند. به طور مشابه 
$P(\bar{t})$
به معنای احتمال عدم این کلمه است؛ یعنی آنکه چه تعدادی از اسناد شامل این کلمه نیستند.
$P(C_i|t)$
احتمال کلاس
$C_i$
به شرط کلمه
$t$
است؛ بدین معنا که چه تعدادی از اسناد شامل کلمه
$t$
به کلاس
$C_i$
تعلق دارند.
به طور مشابه
$P(C_i|\bar{t})$
هم تعریف می‌شود.

\subsection{شاخص جینی}
 شاخص جینی\LTRfootnote{Gini index}
معیاری دیگر برای انتخاب ویژگی است که در مقالاتی مورد استفاده قرار گرفته است
\cite{labani2018novel}\cite{uysal2016improved}.
نحوه محاسبه این معیار در رابطه ۲-۲ آورده شده است.

\begin{equation}
GI(t) = \sum_{i=1}^M P(t|C_i)^2 P(C_i|t)^2
\end{equation}

در این رابطه 
$GI(t)$
به معنای مقدار شاخص جینی برای کلمه
$t$
است. 
$P(t|C_i)$
احتمال شرطی کلمه 
$t$
نسبت به کلاس
$C_i$
است؛ بدین تعریف که بررسی می‌کند که چه تعداد از اسناد متعلق به کلاس 
$C_i$
دارای کلمه
$t$
هستند. سایر نماد‌های این رابطه در بخش قبل تعریف شده است.

\subsection{نسبت نابرابری}
نسبت نابرابری
\LTRfootnote{Odds Ration}
معیاری است که برای انتخاب ویژگی در مقاله اویسال
\LTRfootnote{Uysal}
استفاده شده است
\cite{uysal2016improved}
. نحوه محاسبه این معیار در رابطه ۲-۳ آورده شده است.

\begin{equation}
OR(t, C_i) = \log{\frac{P(t|C_i)[1-P(t|\bar{C_i})]}{[1-P(t|C_i)]P(t|\bar{C_i})}}
\end{equation}

در این رابطه
$OR(t, C_i)$
نسبت نابرابری به ازای کلمه
$t$
و کلاس
$C_i$
محاسبه شده است. در کار تحقیقاتی اویسال برای جلوگیری از صفر شدن مخرج مقدار 0/01 به صورت و مخرج افزوده شده است
\cite{uysal2016improved}
.

\subsection{معیار زائدی کمینه شباهت بیشینه}
معیار زائدی کمینه شباهت بیشینه
\LTRfootnote{Minimal redundancy maximal relevance}
که با نماد 
$mRMR$
یک روش انتخاب ویژگی چند متغیره است که در مقاله لبنی و همکاران مورد استفاده قرار گرفته است
\cite{labani2018novel}
. نحوه محاسبه این معیار در رابطه ۲-۴ آمده است.
\begin{equation}
mRMR(f_j) = I(f_j, C_k) - \frac{1}{|S|-1} \sum_{f_i \in S} I(f_i, f_j)
\end{equation}

در این رابطه مجموعه
$S$
به معنی مجموعه ویژگی‌های انتخابی است.
$I(a, b)$
به معنای اطلاعات متقابل
\LTRfootnote{Mutual information}
$a$
و
$b$
است.

اگر به منطق این رابطه نگاه کنیم، در می‌یابیم با این معیار به دنبال ویژگی‌های هستیم که با داده‌های یک کلاس ارتباط بالایی داشته باشند و با ویژگی‌هایی که در حال حاضر انتخاب شده‌‌اند شباهت پایین.

\subsection{معیار تمایزگر نسبی}
معیار تمایزگر نسبی\LTRfootnote{Relative discriminative criterion} یک روش انتخاب ویژگی برای مسائل دسته‌بندی دودویی است که در مقاله لبنی و همکاران\cite{labani2018novel} مورد استفاده بدوه است. نحوه محاسبه این معیار در رابطه ۲-۵ آمده است.

\begin{equation}
RDC(t, tc_i(t)) = \frac{|df_{pos}(t)-df_{neg}(t)|}{\min(df_{pos}(t),df_{neg}(t)).tc_i(t)}
\end{equation}

در این رابطه
$RDC(t, tc_i(t))$
به معنای امتیاز تمایزگر نسبی یک کلمه
$t$
و سند
$i$
-ام است.
$df_{pos}(t)$
و
$df_{neg}(t)$
به ترتیب به معنای تعداد اسناد کلاس مثبت و کلاس منفی که شامل کلمه
$t$
هستند می‌شود. منظور از
$tc_i(t)$
تعداد دفعات تکرار کلمه
$t$
در سند
$i$
-ام است. برای آنکه بتوان یک امتیاز نهایی به کلمه
$t$
نسبت داد باید تمام این امتیازها را باهم به نوعی جمع زد. مساحت زیر منحنی\LTRfootnote{Area Under the Curve(AUC)} مطابق رابطه ۲-۶ حاصل می‌شود. نهایتا
$AUC(t,tc_i)$
به ازای آخرین سند به عنوان امتیاز نهایی اعلام خواهد شد.

\begin{equation}
\begin{cases}
AUC(t,tc_1) = 0 \\
AUC(t,tc_i) = AUC(t,tc_{i-1}) + \frac{RDC(t,tc_i)+RDC(t,tc_{i+1})}{2}
\end{cases}
\end{equation}

\section{دسته‌بندی روش‌های انتخاب ویژگی}
\section{الگوریتم ژنتیک}
