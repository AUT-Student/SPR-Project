\chapter{راهنمای استفاده از الگوی لاتک دانشگاه صنعتی امیرکبیر(پلی‌تکنیک تهران)}

\section{مقدمه}
حروف‌چینی پروژه کارشناسی، پایان‌نامه یا رساله یکی از موارد پرکاربرد استفاده از زی‌پرشین است. از طرفی، یک پروژه، پایان‌نامه یا رساله،  احتیاج به تنظیمات زیادی از نظر صفحه‌آرایی  دارد که ممکن است برای
یک کاربر مبتدی، مشکل باشد. به همین خاطر، برای راحتی کار کاربر، یک کلاس با نام 
\verb;AUTthesis;
 برای حروف‌چینی پروژه‌ها، پایان‌نامه‌ها و رساله‌های دانشگاه صنعتی امیرکبیر با استفاده از نرم‌افزار زی‌پرشین،  آماده شده است. این فایل به 
گونه‌ای طراحی شده است که کلیه خواسته‌های مورد نیاز  مدیریت تحصیلات تکمیلی دانشگاه صنعتی امیرکبیر را برآورده می‌کند و نیز، حروف‌چینی بسیاری
از قسمت‌های آن، به طور خودکار انجام می‌شود.

کلیه فایل‌های لازم برای حروف‌چینی با کلاس گفته شده، داخل پوشه‌ای به نام
\verb;AUTthesis;
  قرار داده شده است. توجه داشته باشید که برای استفاده از این کلاس باید فونت‌های
  \verb;Nazanin B;،
 \verb;PGaramond;
 و
  \verb;IranNastaliq;
    روی سیستم شما نصب شده باشد.
\section{این همه فایل؟!}\label{sec2}
از آنجایی که یک پایان‌نامه یا رساله، یک نوشته بلند محسوب می‌شود، لذا اگر همه تنظیمات و مطالب پایان‌نامه را داخل یک فایل قرار بدهیم، باعث شلوغی
و سردرگمی می‌شود. به همین خاطر، قسمت‌های مختلف پایان‌نامه یا رساله  داخل فایل‌های جداگانه قرار گرفته است. مثلاً تنظیمات پایه‌ای کلاس، داخل فایل
\verb;AUTthesis.cls;، 
تنظیمات قابل تغییر توسط کاربر، داخل 
\verb;commands.tex;،
قسمت مشخصات فارسی پایان‌نامه، داخل 
\verb;fa_title.tex;,
مطالب فصل اول، داخل 
\verb;chapter1;
و ... قرار داده شده است. نکته مهمی که در اینجا وجود دارد این است که از بین این  فایل‌ها، فقط فایل 
\verb;AUTthesis.tex;
قابل اجرا است. یعنی بعد از تغییر فایل‌های دیگر، برای دیدن نتیجه تغییرات، باید این فایل را اجرا کرد. بقیه فایل‌ها به این فایل، کمک می‌کنند تا بتوانیم خروجی کار را ببینیم. اگر به فایل 
\verb;AUTthesis.tex;
دقت کنید، متوجه می‌شوید که قسمت‌های مختلف پایان‌نامه، توسط دستورهایی مانند 
\verb;input;
و
\verb;include;
به فایل اصلی، یعنی 
\verb;AUTthesis.tex;
معرفی شده‌اند. بنابراین، فایلی که همیشه با آن سروکار داریم، فایل 
\verb;AUTthesis.tex;
است.
در این فایل، فرض شده است که پایان‌نامه یا رساله شما، از5 فصل و یک پیوست، تشکیل شده است. با این حال، اگر
  پایان‌نامه یا رساله شما، بیشتر از 5 فصل و یک پیوست است، باید خودتان فصل‌های بیشتر را به این فایل، اضافه کنید. این کار، بسیار ساده است. فرض کنید بخواهید یک فصل دیگر هم به پایان‌نامه، اضافه کنید. برای این کار، کافی است یک فایل با نام 
\verb;chapter6;
و با پسوند 
\verb;.tex;
بسازید و آن را داخل پوشه 
\verb;AUTthesis;
قرار دهید و سپس این فایل را با دستور 
\texttt{\textbackslash include\{chapter6\}}
داخل فایل
\verb;AUTthesis.tex;
و بعد از دستور
\texttt{\textbackslash include\{chapter6\}}
 قرار دهید.

\section{از کجا شروع کنم؟}
قبل از هر چیز، بدیهی است که باید یک توزیع تِک مناسب مانند 
\verb;Live TeX;
و یک ویرایش‌گر تِک مانند
\verb;Texmaker;
را روی سیستم خود نصب کنید.  نسخه بهینه شده 
\verb;Texmaker;
را می‌توانید  از سایت 
 \href{http://www.parsilatex.com}{پارسی‌لاتک}%
\LTRfootnote{\url{http://www.parsilatex.com}}
 و
\verb;Live TeX;
را هم می‌توانید از 
 \href{http://www.tug.org/texlive}{سایت رسمی آن}%
\LTRfootnote{\url{http://www.tug.org/texlive}}
 دانلود کنید.
 
در مرحله بعد، سعی کنید که  یک پشتیبان از پوشه 
\verb;AUTthesis;
 بگیرید و آن را در یک جایی از هارددیسک سیستم خود ذخیره کنید تا در صورت خراب کردن فایل‌هایی که در حال حاضر، با آن‌ها کار می‌کنید، همه چیز را از 
 دست ندهید.
 
 حال اگر نوشتن پایان‌نامه اولین تجربه شما از کار با لاتک است، توصیه می‌شود که یک‌بار به طور سرسری، کتاب «%
\href{http://www.tug.ctan.org/tex-archive/info/lshort/persian/lshort.pdf}{مقدمه‌ای نه چندان کوتاه بر
\lr{\LaTeXe}}\LTRfootnote{\url{http://www.tug.ctan.org/tex-archive/info/lshort/persian/lshort.pdf}}»
   ترجمه دکتر مهدی امیدعلی، عضو هیات علمی دانشگاه شاهد را مطالعه کنید. این کتاب، کتاب بسیار کاملی است که خیلی از نیازهای شما در ارتباط با حروف‌چینی را برطرف می‌کند.
 
 
بعد از موارد گفته شده، فایل 
\verb;AUTthesis.tex;
و
\verb;fa_title;
را باز کنید و مشخصات پایان‌نامه خود مثل نام، نام خانوادگی، عنوان پایان‌نامه و ... را جایگزین مشخصات موجود در فایل
\verb;fa_title;
 کنید. دقت داشته باشید که نیازی نیست 
نگران چینش این مشخصات در فایل پی‌دی‌اف خروجی باشید. فایل 
\verb;AUTthesis.cls;
همه این کارها را به طور خودکار برای شما انجام می‌دهد. در ضمن، موقع تغییر دادن دستورهای داخل فایل
\verb;fa_title;
 کاملاً دقت کنید. این دستورها، خیلی حساس هستند و ممکن است با یک تغییر کوچک، موقع اجرا، خطا بگیرید. برای دیدن خروجی کار، فایل 
\verb;fa_title;
 را 
\verb;Save;، 
(نه 
\verb;As Save;)
کنید و بعد به فایل 
\verb;AUTthesis.tex;
برگشته و آن را اجرا کنید. حال اگر می‌خواهید مشخصات انگلیسی پایان‌نامه را هم عوض کنید، فایل 
\verb;en_title;
را باز کنید و مشخصات داخل آن را تغییر دهید.%
\RTLfootnote{
برای نوشتن پروژه کارشناسی، نیازی به وارد کردن مشخصات انگلیسی پروژه نیست. بنابراین، این مشخصات، به طور خودکار،
نادیده گرفته می‌شود.
}
 در اینجا هم برای دیدن خروجی، باید این فایل را 
\verb;Save;
کرده و بعد به فایل 
\verb;AUTthesis.tex;
برگشته و آن را اجرا کرد.

برای راحتی بیشتر، 
فایل 
\verb;AUTthesis.cls;
طوری طراحی شده است که کافی است فقط  یک‌بار مشخصات پایان‌نامه  را وارد کنید. هر جای دیگر که لازم به درج این مشخصات باشد، این مشخصات به طور خودکار درج می‌شود. با این حال، اگر مایل بودید، می‌توانید تنظیمات موجود را تغییر دهید. توجه داشته باشید که اگر کاربر مبتدی هستید و یا با ساختار فایل‌های  
\verb;cls;
 آشنایی ندارید، به هیچ وجه به این فایل، یعنی فایل 
\verb;AUTthesis.cls;
دست نزنید.

نکته دیگری که باید به آن توجه کنید این است که در فایل 
\verb;AUTthesis.cls;،
سه گزینه به نام‌های
\verb;bsc;,
\verb;msc;
و
\verb;phd;
برای تایپ پروژه، پایان‌نامه و رساله،
طراحی شده است. بنابراین اگر قصد تایپ پروژه کارشناسی، پایان‌نامه یا رساله را دارید، 
 در فایل 
\verb;AUTthesis.tex;
باید به ترتیب از گزینه‌های
\verb;bsc;،
\verb;msc;
و
\verb;phd;
استفاده کنید. با انتخاب هر کدام از این گزینه‌ها، تنظیمات مربوط به آنها به طور خودکار، اعمل می‌شود.

\section{مطالب پایان‌نامه را چطور بنویسم؟}
\subsection{نوشتن فصل‌ها}
همان‌طور که در بخش 
\ref{sec2}
گفته شد، برای جلوگیری از شلوغی و سردرگمی کاربر در هنگام حروف‌چینی، قسمت‌های مختلف پایان‌نامه از جمله فصل‌ها، در فایل‌های جداگانه‌ای قرار داده شده‌اند. 
بنابراین، اگر می‌خواهید مثلاً مطالب فصل ۱ را تایپ کنید، باید فایل‌های 
\verb;AUTthesis.tex;
و
\verb;chapter1;
را باز کنید و محتویات داخل فایل 
\verb;chapter1;
را پاک کرده و مطالب خود را تایپ کنید. توجه کنید که همان‌طور که قبلاً هم گفته شد، تنها فایل قابل اجرا، فایل 
\verb;AUTthesis.tex;
است. لذا برای دیدن حاصل (خروجی) فایل خود، باید فایل  
\verb;chapter1;
را 
\verb;Save;
کرده و سپس فایل 
\verb;AUTthesis.tex;
را اجرا کنید. یک نکته بدیهی که در اینجا وجود دارد، این است که لازم نیست که فصل‌های پایان‌نامه را به ترتیب تایپ کنید. می‌توانید ابتدا مطالب فصل ۳ را تایپ کنید و سپس مطالب فصل ۱ را تایپ کنید.

نکته بسیار مهمی که در اینجا باید گفته شود این است که سیستم
\lr{\TeX},
محتویات یک فایل تِک را به ترتیب پردازش می‌کند. به عنوان مثال، اگه فایلی، دارای ۴ خط دستور باشد، ابتدا خط ۱، بعد خط ۲، بعد خط ۳ و در آخر، خط ۴ پردازش می‌شود. بنابراین، اگر مثلاً مشغول تایپ مطالب فصل ۳ هستید، بهتر است
که دو دستور
\verb~\chapter{مقدمه}
~
و
\verb~\chapter{مفاهیم تئوری}
در این بخش قصد داریم در مورد مفاهیم تئوری که در روش‌های مورد بررسی این پروژه استفاده شده‌اند بپردازیم. به طور دقیق‌تر ابتدا در مورد روش‌های انتخاب ویژگی معروف صحبت خواهد شد و سپس الگوریتم ژنتیک توضیح داده می‌شود.


\section{دسته‌بندی روش‌های انتخاب ویژگی}
روش‌های انتخاب ویژگی به چندین دسته تقسیم می‌شوند. دو روش متداول و شناخته‌شده‌تر آن روش‌های فیلتر\LTRfootnote{Filter} و پوشاننده\LTRfootnote{Wrapper} هستند. در روش‌های پوشاننده مستقیما ویژگی‌های انتخاب‌شده را در یک مسئله واقعی (که در اینجا یک مسئله دسته‌بندی متن است) استفاده می‌کنند و لذا امتیازی که برای یک مجموعه ویژگی انتخاب‌شده بدست می‌آید امتیاز دقت واقعی برای مسئله دسته‌بندی است. در مقابل و در روش‌های فیلتر،‌ با اعمال روش‌های آماری سعی می‌شود که یک امتیاز برای یک مجموعه ویژگی انتخاب‌شده حاصل گردد.
\\

روش‌های پوشاننده چون به صورت مستقیم مجموعه ویژگی را بررسی می‌کند منجر به خروجی دقیق‌تری می‌شود؛ اما باید توجه داشت که روش‌های فیلتر زمان اجرای به مراتب بهتری دارند و بر روی مسئله دسته‌بندی بایاس نخواهند شد\cite{labani2018novel}. در مسائل دسته‌بندی چون تعداد ویژگی‌ها بسیار بالاست نمی‌توان از روش‌های پوشاننده مستقیم استفاده کرد و لذا یا باید از روش‌های فیلتر استفاده کرد و یا به صورت ترکیبی از این دو شیوه بهره گرفت.
\\

روش‌های فیلتر خود به چندین دسته قابل تقسیم هستند؛ نخست آنکه می‌توان این روش‌ها را به روش‌های محلی\LTRfootnote{Local} و روش‌های جهانی\LTRfootnote{Global} تقسیم کرد. در روش‌های جهانی به ویژگی یک امتیاز مطلق داده می‌شود، اما در روش‌های محلی به هر ویژگی متناسب با هر کلاس یک امتیاز داده می‌شود؛ یعنی در روش‌های محلی مشخص است که یک ویژگی برای هر کلاس تا چه میزان خاصیت متمایزکننده دارد. در حالتی که از یک معیار محلی استفاده می‌شود می‌توان مشخص کرد که یک ویژگی می‌توان عضویت یک متن به یک کلاس را نشان دهد یا آنکه عدم عضویت را می‌تواند به خوبی نشان دهد. اگر عضویت را بتواند بهتر نشان دهد آن را یک ویژگی مثبت\LTRfootnote{Positive} و در غیر این صورت آن را یک ویژگی منفی\LTRfootnote{Negative} برای آن کلاس به حساب می‌آورند.\cite{uysal2016improved}.
\\

روش‌های فیلتر را می‌توان به دو دسته تک متغیره\LTRfootnote{Univariate} و چندمتغیره \LTRfootnote{Multivariate} هم تقسیم کرد. در روش‌های تک متغیره هر ویژگی به صورت مستقل از سایر ویژگی‌ها امتیاز دریافت می‌کند، ولی در روش‌های چند متغیره در کنار آن که به شباهت ویژگی به هدف نگاه می‌شود، به زائد نبودن ویژگی‌ها نسبت به یکدیگر هم توجه می‌شود.\cite{labani2018novel}

\section{محاسبه روش‌های انتخاب ویژگی}
در این بخش مهم‌ترین معیار‌های انتخاب ویژگی معرفی می‌شوند و نحوه محاسبه آن‌ها ارائه می‌شود. این معیار‌ها تماما جز روش‌های انتخاب ویژگی فیلتر هستند.

\subsection{بهره اطلاعاتی}
بهره اطلاعاتی\LTRfootnote{Information Gain}
یکی از معیارهای محبوب برای انتخاب ویژگی در مقالات است
\cite{labani2018novel}\cite{uysal2016improved}\cite{ghareb2016hybrid}.
نحوه محاسبه این معیار برای یک کلمه در رابطه ۲-۱ آمده است.

\begin{equation}
IG(t) = -\sum_{i=1}^M P(C_i)\log{P(C_i)} + P(t)\sum_{i=1}^M P(C_i|t)\log{P(C_i|t)} + P(\bar{t})\sum_{i=1}^M P(C_i|\bar{t})\log{P(C_i|\bar{t})}
\end{equation}

در این رابطه
$IG(t)$
به معنای مقدار بهره اطلاعاتی برای کلمه
$t$
است. 
$M$
  برابر با تعداد کلاس‌ها است.
$P(C_i)$
احتمال کلاس
$C_i$
است؛ یعنی چه تعدادی از اسناد به این کلاس تعلق دارند.
$P(t)$
احتمال مربوط به کلمه
$t$
است؛ یعنی آنکه چه تعدادی از اسناد شامل این کلمه هستند. به طور مشابه 
$P(\bar{t})$
به معنای احتمال عدم این کلمه است؛ یعنی آنکه چه تعدادی از اسناد شامل این کلمه نیستند.
$P(C_i|t)$
احتمال کلاس
$C_i$
به شرط کلمه
$t$
است؛ بدین معنا که چه تعدادی از اسناد شامل کلمه
$t$
به کلاس
$C_i$
تعلق دارند.
به طور مشابه
$P(C_i|\bar{t})$
هم تعریف می‌شود.

\subsection{شاخص جینی}
 شاخص جینی\LTRfootnote{Gini index}
معیاری دیگر برای انتخاب ویژگی است که در مقالاتی مورد استفاده قرار گرفته است
\cite{labani2018novel}\cite{uysal2016improved}.
نحوه محاسبه این معیار در رابطه ۲-۲ آورده شده است.

\begin{equation}
GI(t) = \sum_{i=1}^M P(t|C_i)^2 P(C_i|t)^2
\end{equation}

در این رابطه 
$GI(t)$
به معنای مقدار شاخص جینی برای کلمه
$t$
است. 
$P(t|C_i)$
احتمال شرطی کلمه 
$t$
نسبت به کلاس
$C_i$
است؛ بدین تعریف که بررسی می‌کند که چه تعداد از اسناد متعلق به کلاس 
$C_i$
دارای کلمه
$t$
هستند. سایر نماد‌های این رابطه در بخش قبل تعریف شده است.

\subsection{نسبت نابرابری}
نسبت نابرابری
\LTRfootnote{Odds Ration}
معیاری است که برای انتخاب ویژگی در مقاله اویسال
\LTRfootnote{Uysal}
  و مقاله غارب و همکاران \LTRfootnote{Ghareb} استفاده شده است
\cite{uysal2016improved}\cite{ghareb2016hybrid}
. نحوه محاسبه این معیار در رابطه ۲-۳ آورده شده است.

\begin{equation}
OR(t, C_i) = \log{\frac{P(t|C_i)[1-P(t|\bar{C_i})]}{[1-P(t|C_i)]P(t|\bar{C_i})}}
\end{equation}

در این رابطه
$OR(t, C_i)$
نسبت نابرابری به ازای کلمه
$t$
و کلاس
$C_i$
محاسبه شده است. در کار تحقیقاتی اویسال برای جلوگیری از صفر شدن مخرج مقدار 0/01 به صورت و مخرج افزوده شده است
\cite{uysal2016improved}
.

\subsection{معیار زائدی کمینه شباهت بیشینه}
معیار زائدی کمینه شباهت بیشینه
\LTRfootnote{Minimal redundancy maximal relevance}
که با نماد 
$mRMR$
یک روش انتخاب ویژگی چند متغیره است که در مقاله لبنی و همکاران مورد استفاده قرار گرفته است
\cite{labani2018novel}
. نحوه محاسبه این معیار در رابطه ۲-۴ آمده است.
\begin{equation}
mRMR(f_j) = I(f_j, C_k) - \frac{1}{|S|-1} \sum_{f_i \in S} I(f_i, f_j)
\end{equation}

در این رابطه مجموعه
$S$
به معنی مجموعه ویژگی‌های انتخابی است.
$I(a, b)$
به معنای اطلاعات متقابل
\LTRfootnote{Mutual information}
$a$
و
$b$
است.

اگر به منطق این رابطه نگاه کنیم، در می‌یابیم با این معیار به دنبال ویژگی‌های هستیم که با داده‌های یک کلاس ارتباط بالایی داشته باشند و با ویژگی‌هایی که در حال حاضر انتخاب شده‌‌اند شباهت پایین.

\subsection{معیار تمایزگر نسبی}
معیار تمایزگر نسبی\LTRfootnote{Relative discriminative criterion} یک روش انتخاب ویژگی برای مسائل دسته‌بندی دودویی است که در مقاله لبنی و همکاران\cite{labani2018novel} مورد استفاده شده است. نحوه محاسبه این معیار در رابطه ۲-۵ آمده است.

\begin{equation}
RDC(t, tc_i(t)) = \frac{|df_{pos}(t)-df_{neg}(t)|}{\min(df_{pos}(t),df_{neg}(t)).tc_i(t)}
\end{equation}

در این رابطه
$RDC(t, tc_i(t))$
به معنای امتیاز تمایزگر نسبی یک کلمه
$t$
و سند
$i$
-ام است.
$df_{pos}(t)$
و
$df_{neg}(t)$
به ترتیب به معنای تعداد اسناد کلاس مثبت و کلاس منفی که شامل کلمه
$t$
هستند می‌شود. منظور از
$tc_i(t)$
تعداد دفعات تکرار کلمه
$t$
در سند
$i$
-ام است. برای آنکه بتوان یک امتیاز نهایی به کلمه
$t$
نسبت داد باید تمام این امتیازها را باهم به نوعی جمع زد. مساحت زیر منحنی\LTRfootnote{Area Under the Curve(AUC)} مطابق رابطه ۲-۶ حاصل می‌شود. نهایتا
$AUC(t,tc_i)$
به ازای آخرین سند به عنوان امتیاز نهایی اعلام خواهد شد.

\begin{equation}
\begin{cases}
AUC(t,tc_1) = 0 \\
AUC(t,tc_i) = AUC(t,tc_{i-1}) + \frac{RDC(t,tc_i)+RDC(t,tc_{i+1})}{2}
\end{cases}
\end{equation}

\section{الگوریتم ژنتیک}
الگوریتم ژنتیک\LTRfootnote{Genetic algorithm} یک الگوریتم تکاملی\LTRfootnote{Evolutionary algorithm} است که با اقتباس از فرآیند تکامل موجودات زنده ارائه شده است. از آنجایی که این الگوریتم قسمت اصلی مقاله غارب و همکاران\cite{ghareb2016hybrid} را تشکیل می‌دهد، در این قسمت به صورت مختصر توضیح داده می‌شود.
\\

در الگوریتم ژنتیک ابتدا باید هر جوابی که برای مسئله وجود دارد را در قالب یک وضعیت بازنمایی\LTRfootnote{Representation} کرد. در این حالت هر وضعیت نقش کرومزوم یک شخص را خواهد داشت و ژن‌های این کرومزوم مرتبط با جزئیات آن وضعیت است. سپس باید یک تعداد زیادی فرد با کرومزوم اولیه ایجاد کرد؛ چیزی که به آن جمعیت اولیه\LTRfootnote{Initial population} گفته می‌شود. در الگوریتم‌های ژنتیک لازم است تا یک تابع شایستگی\LTRfootnote{Fitness function} تعریف شود. فردی که شایستگی بیشتری دارد باید مطابق با قانون تکامل شانس بیشتری برای زنده ماندن و تکثیر نسل داشته باشد. این چیزی است که در گام انتخاب والدین رخ می‌دهد. در گام انتخاب والدین، افراد با شایستگی بیشتر شانس انتخاب بیشتری دارند. سپس هر دو والد دو فرزند را ایجاد می‌کنند که ژن این دو حاصل ترکیب ژن والدین است. نحوه ترکیب ژن والدین و ایجاد ژن فرزندان را بازترکیب\LTRfootnote{Crossover} گویند. نهایتا باید عمل جهش\LTRfootnote{Mutation} هم تعریف شود. در جهش برخی از ژن‌های تعداد کمی از افراد تغییر می‌کند. پس از آنکه نسل جدید به وجود آمدند، نسل پیشین از بین می‌رود و الگوریتم ژنتیک با نسل جدید ادامه پیدا می‌کند تا جایی که یک شرط خاتمه برقرار شود. این شرط خاتمه می‌تواند تعداد نسل مشخص و یا همگرایی نسل‌ها باشد.
\\

در ابتدای یک الگوریتم ژنتیک عملا تعدادی جواب تصادفی اولیه برای مسئله داریم و در حین الگوریتم با نسل‌های جدید، جواب‌های موجود هم بهتر می‌شود؛ چراکه یک جواب مناسب در صورتی که تابع شایستگی به خوبی تعریف شده باشد، منجر به ایجاد جواب‌های بیشتری مبتنی بر خود می‌شود و جواب‌ها نامناسب کنار گذاشته می‌شوند. نهایتا آنکه عمل بازترکیب و جهش می‌توانند تنوع جواب‌ها را حفظ کنند و به وضعیت‌هایی برسیم که در ابتدا قابل ساختن نبوده است. برای آنکه یک الگوریتم برپایه ژنتیک معرفی شود لازم است تا گام‌های گفته شده طراحی شوند؛ یعنی به عنوان مثال مشخص باشد که عمل بازترکیب چگونه رخ می‌دهد.~
را در فایل 
\verb~AUTthesis.tex~،
غیرفعال%
\RTLfootnote{
برای غیرفعال کردن یک دستور، کافی است پشت آن، یک علامت
\%
 بگذارید.
}
 کنید. زیرا در غیر این صورت، ابتدا مطالب فصل ۱ و ۲ پردازش شده (که به درد ما نمی‌خورد؛ چون ما می‌خواهیم خروجی فصل ۳ را ببینیم) و سپس مطالب فصل ۳ پردازش می‌شود و این کار باعث طولانی شدن زمان اجرا می‌شود. زیرا هر چقدر حجم فایل اجرا شده، بیشتر باشد، زمان بیشتری هم برای اجرای آن، صرف می‌شود.

\subsection{مراجع}
برای وارد کردن مراجع به فصل 2
مراجعه کنید.
\subsection{واژه‌نامه فارسی به انگلیسی و برعکس}
برای وارد کردن واژه‌نامه فارسی به انگلیسی و برعکس، بهتر است مانند روش بکار رفته در فایل‌های 
\verb;dicfa2en;
و
\verb;dicen2fa;
عمل کنید.

\section{اگر سوالی داشتم، از کی بپرسم؟}
برای پرسیدن سوال‌های خود در مورد حروف‌چینی با زی‌پرشین،  می‌توانید به
 \href{http://forum.parsilatex.com}{تالار گفتگوی پارسی‌لاتک}%
\LTRfootnote{\url{http://www.forum.parsilatex.com}}
مراجعه کنید. شما هم می‌توانید روزی به سوال‌های دیگران در این تالار، جواب بدهید.
