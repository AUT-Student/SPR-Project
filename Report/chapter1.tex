\chapter{مقدمه}
بالای ۸۰ درصد از اطلاعات موجود در قالب داده‌های متنی ذخیره شده‌اند\cite{ghareb2016hybrid}. پردازش این داده‌ها در حوزه پردازش زبان طبیعی\LTRfootnote{Natural language processing(NLP)} است. یکی از کاربردهای این حوزه دسته‌بندی متون\LTRfootnote{Text Classification} به تعدادی دسته از پیش تعیین شده است؛ به عنوان مثال ایمل‌های دریافتی یک فرد را در نظر بگیرید. تعدادی از ایمیل‌ها، ایمیل‌هایی هستند که کاربر مایل به دریافت آن است و تعدادی دیگر هرزنامه\LTRfootnote{Spam} هستند. طبیعی است که کاربران دوست نداشته باشند که صندوق دریافتی آن‌ها شامل هرزنامه‌ها شود؛ پس در این شرایط نیاز به سیستمی است که پیام‌های متنی ورودی را به دو کلاس تقسیم کند. تشخیص هرزنامه‌ها شاید یکی از معروف‌ترین کاربرد‌های دسته‌بندی متن باشد اما قطعا تنها کاربرد آن نیستند!
\\

هر متن دارای ویژگی‌هایی است؛ این ویژگی‌ها در روش‌های دسته‌بندی مختلف استفاده می‌شوند و به سبب آن‌ها امکان توسعه یک مدل دسته‌بند متن وجود خواهد داشت. برای بدست آوردن ویژگی‌های یک متن روش‌های گوناگونی وجود دارد. یکی از روش‌ها تهیه بردارهایی از متون است که هر بعد آن متناسب با یکی از کلمات موجود در دیکشنری باشد. بدین شکل که اگر متنی تعداد زیادی از کلمه اول را در خود داشته باشد، مقدار بعد اولش زیاد خواهد بود و بالعکس. این روش اگرچه در مقایسه با روش‌های عصبی روش جدیدی محسوب نمی‌شود ولی با این حال چندان قدیمی هم نیست و همچنان در شرایطی که داده‌‌ی کافی وجود نداشته باشد قابل استفاده هستند. همانطور که گفته شد در این روش‌ها به ازای هر کلمه در لغتنامه، یک ویژگی\LTRfootnote{Feature} درنظر گرفته می‌شود و بدین ترتیب ابعاد فضای مسئله بسیار بالا خواهد بود. ابعاد بالای مسئله باعث خواهد شد که روش‌های مرسوم برای دسته‌بندی دچار مشکل جدی شوند.
\\

برای حل مشکل ابعاد بالا یک راه حل استفاده از روش‌های انتخاب ویژگی\LTRfootnote{Feature Selection} است. در روش‌های انتخاب ویژگی متناسب با شرایط مسئله تعدادی از ویژگی‌ها انتخاب می‌شوند و مابقی ویژگی‌ها حذف می‌شوند. بدین ترتیب در یک فضای با ابعاد کمتر و ویژگی کمتر با سهولت بیشتر می‌تواند روش‌های دسته‌بندی متن را استفاده کرد. سوالی که باید به آن جواب داد این است که «چگونه می‌توان ویژگی‌های یک مسئله دسته‌بندی متن را انتخاب کرد؟»
\\

در این پروژه تحقیقاتی، من سه روش جدید و معتبر \cite{uysal2016improved} \cite{labani2018novel} \cite{ghareb2016hybrid}که برای انتخاب ویژگی در مسائل دسته‌بندی معرفی شده‌اند را تبیین می‌کنم و تفاوت میان آن‌ها را مورد بررسی قرار خواهم داد. بدین ترتیب ابتدا در فصل دوم مفاهیم تئوری که برای درک روش‌های مذکور مورد نیاز است بیان خواهد شد. در فصل سوم و با تکیه به مفاهیم تئوری هر یک از سه روش در یک بخش مجزا تشریح می‌شود. در فصل چهارم ارزیابی و مقایسه‌ای میان سه روش صورت می‌گیرد. نهایتا در فصل پنجم جمع‌بندی و نتیجه‌گیری مطالب گفته‌شده در مقاله ارائه خواهد شد. 